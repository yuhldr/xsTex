% \documentclass{ldrsimple}
\documentclass{../../ppt/ldrsimple}

%------------------------------------------------------------
%首页start
\title{题目}

\subtitle{副标题}

\author[yuh]
{余航}

\institute[ICPCS]
{
机构\\
可以换行
}

\date[\today]{\today}

%首页end
%------------------------------------------------------------

\begin{document}

%封面
\frame{\titlepage}

\section{初次计算}

\subsection*{热浴区别}


\begin{frame}{标题2}

    测试\footfullcite{greaneyAnomalousDissipationSinglewalled2009},\footfullcite{kongPhononDispersionMeasured2011},\footfullcite{tussyadiah2015hotels}

    This is an example with a \href{https://www.example.com}{link}.

    \begin{equation}
        Y=S
    \end{equation}

\end{frame}


\begin{frame}{文本段落等间距}
    这是段落间距

    Linux内核,常见发行版:Ubuntu-LTS、openSUSE-Leap,文本行间距测试查看,文本行间距测试查看,文本行间距测试查看
    \begin{itemize}
        \item 最外层为根目录 /,普通用户不可访问
        \item 所有人数据都在 /home/,比如yuh用户文件在 /home/yuh,用户自己可以通过 $\thicksim/$ 代表自己这个目录,此目录自己有完全权限,无法访问其他人数据
    \end{itemize}

    测试一下图文间距

    \includegraphics[width=0.4\textwidth]{lzu_logo_w.pdf}

\end{frame}


\section{常用示例}

\begin{frame}{表格合并}
    \begin{table}
        \begin{tabular}{c|c|c|c|c|c}
            \toprule 节点                  & 分区                    & 最大运行时间               & CPU线程数               & 内存                     & 购买年份                   \\
            \midrule node1               & \multirow{2}{*}{ptt1} & \multirow{4}{*}{无限制} & \multirow{8}{*}{144} & \multirow{2}{*}{1007G} & \multirow{8}{*}{2018年} \\
            \cline{1-1} node2            &                       &                      &                      &                        &                        \\
            \cline{1-2}\cline{5-5} node3 & \multirow{2}{*}{ptt2} &                      &                      & \multirow{6}{*}{251G}  &                        \\
            \cline{1-1} node4            &                       &                      &                      &                        &                        \\
            \cline{1-3} node5            & \multirow{4}{*}{ptt3} & \multirow{4}{*}{14天} &                      &                        &                        \\
            \cline{1-1} node6            &                       &                      &                      &                        &                        \\
            \cline{1-1} node7            &                       &                      &                      &                        &                        \\
            \cline{1-1} node8            &                       &                      &                      &                        &                        \\
            \bottomrule
        \end{tabular}
    \end{table}

\end{frame}



\begin{frame}{插入代码}
    建议使用 sbatch
    \lstinputlisting[language=Bash]{data/lammps.sh}
\end{frame}

\begin{frame}{图多标题}
文本位置文本位置文本位置文本位置文本位置文本位置文本位置
    \begin{figure}[H]
        \centering
        \caption{总标题}
        \subfloat[分标题1]{
            \includegraphics[width=0.48\textwidth]{lzu_logo_w.pdf}
        }
        \subfloat[分标题2]{
            \includegraphics[width=0.15\textwidth]{lzu_logo.pdf}
        }\\
        \subfloat[分标题3]{
            \includegraphics[width=0.48\textwidth]{lzu_logo_w.pdf}
        }
        \subfloat[分标题3]{
            \includegraphics[width=0.15\textwidth]{lzu_logo.pdf}
        }
    \end{figure}
\end{frame}


\end{document}